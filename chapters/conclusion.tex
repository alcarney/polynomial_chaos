\chapter{Conclusion}\label{chap:conclusion}

In summary, we considered numerous cases of \textit{Laplace's Equation}, first
in Chapters \ref{chap:oned-deterministic} and \ref{chap:twod-deterministic} we
tackled the deterministic problem in one and two dimensions respectively. By
recasting the problem in terms of Sobolev spaces and weak solutions we were
able to to derive the \textit{Finite Element Method} construction using an
appropriate finite dimensional subspace in order to approximate the solution.
The resulting set of linear equations could easily be solved using Python code
and we were able to demonstrate convergence of these approximations against the
known analytic solution.

The results from the deterministic cases formed a starting point from which we
were able to build from when it came to tackling the stochastic cases of the
equation. Following similar ideas as those used when discretising the physical
domain of the problem, the \textit{Generalised Polynomial Chaos} provided
results which allowed us to determine an appropriate basis with which we could
define a finite dimensional subspace of the probability space.

In Chapters \ref{chap:oned-stochastic} and \ref{chap:twod-stochastic} we put
these ideas to use and were able to reduce the stochastic problems to much
larger, but still deterministic systems of equations which again we could solve
using Python code. We were also able to calculate a few basic properties of the
resulting solution process, namely the mean and variance.

\subsection*{Further Work}

This project merely scratched the surface of what is a very rich subject and
there is plenty of scope for the ideas presented here to be explored further.
Even if we ignore the application of these methods to different equations there
are many directions in which one could explore.

The \textit{Generalised Polynomial Chaos} being a very flexible method means we
could consider cases where not only coefficients in the equation have a degree
of uncertainty but terms on the right hand side or even the boundary conditions
themselves could all be modelled with some uncertainty. Furthermore we only
considered the \textit{Legendre Chaos} where we assumed that random processes
were uniformly distributed but many other chaoses exist which can be used to
model Gaussian, Gamma and even discrete distributions such as the Binomial
distribution.

This project focused on very simple domains, namely interval and square domains
but the flexibility of both the \textit{Finite Element Method} and the
\textit{Generalised Polynomail Chaos} mean that the work could be extended to
handle arbitrary domains. Finally if it wasn't for time constraints a more
detailed analysis on the results could have been conducted, for example a good
question might have been ``which is more important for good convergence: the
degree of the polynomials $p$ or the dimensionality of the subspace $d$?'' It
also would have been good to have made some comparisons with other popular
methods such as \textit{Monte Carlo Simulations} to assess the performance of
this method relative to existing practices

Aside from the mathematical considerations there are a few more practical
concerns worth considering. During the course of running the Python code I
found a bottleneck to be in the construction of the linear systems themselves,
improving both the efficiency of the construction code and the solution code
through the use of methods such as \textit{Gauss Sidel Interation} would make
this work much more useful.

Of course without the code to implement the mathematics discussed above any
further considerations are purely academic so the code must also be extended to
handle cases such as arbitrary domains or stochastic boundary conditions.

Finally the code presented here also required the use of \textit{MATLAB}
to compute the K-L Expansion, it would be worth developing a method applicable
from within Python to help improve the portability and ease of use of the code.
