\chapter{Two Dimensional Stochastic Case}

Returning to two dimensions we consider the following stochastic version of
this equation:

\begin{equation}\label{eq:twod-stochastic}
  \begin{aligned}
      -\nabla\cdot\left[\kappa(\v{x};\omega)\cdot\nabla u(\v{x};\omega) \right]
      &= f(\v{x})\, &\v{x} \in D = [-1,1] \times [-1,1]\, \omega \in \Omega \\
      u(\v{x};\omega) &= 0\, &\v{x} \in \partial D = \Gamma\, \omega \in \Omega
  \end{aligned}
\end{equation}

where $\Gamma = (\{-1,1\} \times [-1,1]) \cup ([-1,1] \times \{-1, 1\})$ and
$\Omega$ is a probability space. As with the previous stochasitc case in
Chapter \ref{chap:one-d-stochastic} we will follow a similar process to the
deterministic case as we saw in Chapter \ref{chap:two-d-deterministic} to
discretise the physical space while following methods outlined in
\cite{general-poly-chaos} to construct a finite dimensional approximation to
the probability space.

\section{Weak Formulation}

\todo[inline]{Give real definitions for $V$ and $W$}

Obtaining the weak forumulation, we multiply through with $w \in W$ and
integrate over the domain:

\begin{equation}
    -\int_{\Omega}\int_D(\nabla\cdot\left[
        \kappa(\v{x};\omega)\cdot\nabla u(\v{x};\omega)\right]
    w(\v{x};\omega))\, d\v{x}\, d\Omega =
        \int_{\Omega}\int_Df(\v{x})w(\v{x};\omega)\, d\v{x}\, d\Omega
\end{equation}

If we apply Green's first integral identity we obtain:

\begin{equation}
    -\int_{\Omega}\left(\int_{\Gamma}\underbrace{\frac{\partial}{\partial n}
        \left[\kappa(x;\omega)\cdot\nabla u(x;\omega)\right]}_{=0}\, d\Gamma
      -\int_D \kappa(x;\omega)\cdot\nabla u(x;\omega)\cdot\nabla w(x;\omega)
      w(x;\omega)\, d\v{x}
  \right)\, d\Omega
\end{equation}

where the integral on $\Gamma$ is zero as elements in $W$ are zero on the
boundary. Therefore the weak form of the problem is given by:

\begin{equation}\label{eq:wk-twod-stochastic}
    \int_{\Omega}\int_D\kappa(x;\omega)\cdot\nabla u(x;\omega)\cdot\nabla
    w(x;\omega)\, d\v{x}\, d\Omega =
    \int_{\Omega}\int_D f(\v{x})w(x;\omega)\, d\v{x}\, d\Omega
\end{equation}

where in this sense a solution would be a function $u \in V$ which satisfies
the the above equation $\forall w \in W$

\section{Discrete Formulation}

In order to construct a finite dimensional approximation to
\myref{eq:twod-stochastic} we must discretise both the physical and probability
spaces. For the physical space, we can proceed just as we did in Chapter
\ref{chap:two-d-deterministic} and construct a uniform triangulation taking
into account that the physical domain is now $[-1,1] \times [-1,1]$.

Given a parameter $N \in \mathbb{N}$ we create a
square grid with spacing $h = 1/N$, then at each of the $M := (N+1)^2$
intersections we place a node $\v{x}_i$, $i \in \{0,\ldots, M\}$. Finally we
can split each grid square into 2 triangular elements $T_k$ using the diagonal
of negative slope. Which we can now use to define the discretisation:

\[
    \mathcal{T}_h = \bigcup_{k=1}^{2N^2}T_k
\]

Upon which we can define finite dimensional subspaces of V and W:

\begin{align*}
    V_h &= \{v \in V: v \text{ is linear on } T_k,\ i \in \{0, \ldots, 2N^2\},
                      v \text{ is continuous on } D\} \\
    W_h &= \{w \in W: w \text{ is linear on } T_k,\ i \in \{0, \ldots, 2N^2\},
                      w \text{ is continuous on } D\}
\end{align*}

Just as in the deterministic case we can use the `hat functions' as our basis
and define each $\phi_i$ associated with a node $\v{x}_i$ in terms of the
reference function \myref{eq:two-d-ref-basis-fn} as such
$\phi_i(\v{x}) = \Phi(\v{x} - \v{x}_i)$. Then we can approximate as follows:

\begin{equation}
    f(\v{x}) \approx \sum_{j=0}^Mf_j\phi_j(\v{x})
\end{equation}

where $f_j = f(\v{x}_j)$. This also allows us to approximate the solution
process $u^h$ as follows:

\begin{equation}
    u^h(\v{x};\omega) = \sum_{j=0}^Mu_j(\omega)\phi_j(\v{x})
\end{equation}

where we now have a finite dimensional representation for $u$ in the physical
space but we also need a discrete representation for $u$ in the probability
space.

As in the one dimensional case in Chapter \ref{chap:one-d-stochastic} we will
use members from the \textit{Askey Scheme} of Orthogonal Polynomials to form a
basis which spans the probability space. This then allows us to approximate the
solution process $u$ as:

\begin{equation}
    u^{h,P}(\v{x};\omega) = \sum_{j=0}^M\sum_{s=0}^Pu_{j,s}
        \chi_s(\xi)\phi_j(\v{x})
\end{equation}

\subsection{Karhumen Loeve Expansion}

\subsection{Derivation of the Global System of Equations}
