\chapter{One Dimensional Stochastic Case}

Here we return to the one dimensional version of this equation, but this time
we introduce some form of uncertainty in the coefficients of the equation:

\begin{equation}\label{eq:oned-stochastic}
  \begin{aligned}
    -a(x;\omega)u''(x;\omega) + bu'(x;\omega) + cu(x;\omega)
                     &= f(x)\, &\text { in } D = [0,1]\, \omega \in \Omega \\
    u(x;\omega) &= 0 &\text{ on } \partial D = \{0,1\}\, \omega \in \Omega
  \end{aligned}
\end{equation}

where $\Omega$ is a probability space. In order to handle this case, we follow a
similar method to that we used in Chapter \ref{chap:oned-deterministic} however with
the introduction of the random parameter $\omega$ there is an additional layer of
complexity to deal with, as well as discretising the physical domain we also need to
be able to discretise the probability space using a finite dimensional approximation.
This is where the generalised polynomial chaos comes in.

Using the methodology outlined in \cite{general-poly-chaos} we can use various
polynomials in the \textit{Askey Scheme} to construct a finite dimensional approximate
representations of various random processes which when used in conjunction with the
Finite Element Method we can model the expected behaviour of the solution process
$u(x;\omega)$

\paragraph{An Example}

\begin{equation}
  \begin{aligned}
    -\frac{d}{dx}\left[\kappa(x;\omega)\frac{d}{dx}u(x;\omega)\right] &= f(x)
        & \text{ in } D\ \omega \in \Omega \\
     u(x;\omega) &= 0 &\text{ on } \partial D\, \omega \in \Omega
  \end{aligned}
\end{equation}

\paragraph{Weak Formulation}

\begin{equation}
	-\int_{\Omega}\int_D \frac{d}{dx}\left[\kappa(x;\omega)\frac{d}{dx}u(x;\omega)\right]
           w(x;\omega)\, dx\, dP = \int_{\Omega}\int_D f(x)w(x;\omega)\, dx\, dP
\end{equation}

Proceeding with integration by parts

\begin{align}
	-\int_{\Omega}\left(
      \underbrace{\left[\kappa(x;\omega)\frac{d}{dx}u(x;\omega)w(x;\omega)\right]}_{= 0}
      - \int_D\kappa(x;\omega)\frac{d}{dx}u(x;\omega)\frac{d}{dx}w(x;\omega)
     \right)\, dP &= \int_{\Omega}\int_D f(x)w(x;\omega)\, dx\, dP \\
     \int_{\Omega}\int_D\kappa(x;\omega)
           \frac{d}{dx}u(x;\omega)\frac{d}{dx}w(x;\omega)\, dx\, dP &=
     \int_{\Omega}\int_D f(x)w(x;\omega)\, dx\, dP
\end{align}

Using the Exapansions

\begin{align}
	u^h(x;\omega) &= \sum_{s=1}^P\sum_{r=1}^Nu_{r,s}\phi_r(x)\chi_s(\omega) \\
    w^h(x;\omega) &= \sum_{s=1}^P\sum_{r=1}^N\phi_r(x)\chi_s(\omega) \\
    \kappa(x;\omega) &\approx \mu + \sigma\sum_{l=1}^d\sqrt{\lambda_l}\beta_l(x)\zeta_l(\omega)
\end{align}

and substituting into the weak formulation we get:

\section{Weak Formulation}

\todo[inline]{Make exact all the functional spaces etc.}

Similarly to the deterministic case we first need to obtain the weak formulation
of the problem which now involves taking the inner product over the ``joint space''
of both the physical and random domains:
