\chapter{The Finite Element Method}

\todo[inline]{Give overview of the FEM}

\section{Deriving the Weak Formulation}

Considering the general form of Laplace's equation:

\begin{align}\label{eq:general_laplace}
    \generalLaplaceEqn \text{, in } \Omega \\
    u &= 0 \text{, on } \partial\Omega = \Gamma
\end{align}

In order to discretise the domain, we first need the weak formulation. To do
this we multiply (\ref{eq:general_laplace}) through with an appropriate test
function from the test space and integrate over the domain.

So for $v \in V$:

\[
    \int_{\Omega}{v(\nabla\cdot(a\nabla{u}))}\,d\Omega +
    \int_{\Omega}\v{b}\cdot v\nabla{u}\,d\Omega +
    \int_{\Omega}cuv\,d\Omega = \int_{\Omega}fv\,d\Omega
\]

Applying Green's first integral identity to the first term we get:

\[
    \int_{\Omega}a\nabla{v}\cdot\nabla{u}\,d\Omega -
    \int_{\Gamma}av
      \underbrace{\frac{\partial{u}}{\partial{n}}}_{=0}
      \,d\Gamma +
    \int_{\Omega}\v{b}\cdot v\nabla{u}\,d\Omega + \int_{\Omega}cuv\,d\Omega =
    \int_{\Omega}fv\,d\Omega
\]

As per our boundary conditions $u=0$ on $\Gamma$ which caues the second
integral to vanish hence:

\[
    \int_{\Omega}a\nabla{v}\cdot\nabla{u}\,d\Omega +
    \int_{\Omega}\v{b}\cdot v\nabla{u}\,d\Omega + \int_{\Omega}cuv\,d\Omega =
    \int_{\Omega}fv\,d\Omega
\]

Then if we define our bilinear form to be:

\[
    a(u,v) =
    \int_{\Omega}a\nabla{v}\cdot\nabla{u}\,d\Omega +
    \int_{\Omega}\v{b}\cdot v\nabla{u}\,d\Omega + \int_{\Omega}cuv\,d\Omega
\]

and the linear functional to be:

\[
    l(v) = \int_{\Omega}fv\,d\Omega
\]

Then we have a weak solution when we can find $u \in V$ such that:

\begin{equation}\label{eq:weak_formulation}
    a(u,v) = l(v)\, \forall v \in V
\end{equation}

\section{The Master Triangle}

As we will see it will be useful to introduce a mapping that will let us
evaluate integrals for arbitrary triangles by considering a single reference
triangle comprised of the points $(0,0), (0,1), (1,0)$

\todo[inline]{Explain the master triangle, the mapping etc}

\begin{align}\label{eq:master_basis_functions}
    \psi_1(\zeta, \eta) &= 1 - \zeta - \eta \\
    \psi_2(\zeta, \eta) &= \zeta \\
    \psi_3(\zeta, \eta) &= \eta
\end{align}

It will also be useful to consider the derivatives of these functions:

\begin{align}\label{eq:master_basis_functions_derivative}
    \psi_1' = \nabla\psi_1(\zeta, \eta) &=
        \frac{1}{h}\left(\begin{array}{c}-1 \\ -1\end{array}\right) \\
    \psi_2' = \nabla\psi_2(\zeta, \eta) &=
        \frac{1}{h}\left(\begin{array}{c}1 \\ 0\end{array}\right) \\
    \psi_3' = \nabla\psi_3(\zeta, \eta) &=
        \frac{1}{h}\left(\begin{array}{c}0 \\ 1\end{array}\right)
\end{align}

\section{The Local Stiffness Matrix}

Each local stiffness matrix will have dimension $3 \times 3$ and will be given
by:

\begin{align*}
    A^k_{i,j} &= a_k(\phi_j, \phi_i) \\
     &= \int_{T_k}a\nabla\phi_i\cdot\nabla\phi_j\,dxdy +
        \int_{T_k}\v{b}\cdot\phi_j\nabla\phi_i\,dxdy +
        \int_{T_k}c\phi_i\phi_j\,dxdy
\end{align*}

To simplify the integrations we will apply our mapping so that we have:

\[
    A^k_{i,j} = h^2\int_Ta\nabla\psi_i\cdot\nabla\psi_j\,d\zeta d\eta +
                h^2\int_T\v{b}\psi_j\nabla\psi_i\,d\zeta d\eta +
                h^2\int_Tc\psi_i\psi_j\,d\zeta d\eta
\]

For simplicity we will consider each integral in turn:

\[
    ah^2\int_T\nabla\psi_i\cdot\nabla\psi_j\,d\zeta d\eta =
    a\int_T\psi'_{i\zeta}\psi'_{j\zeta} +
        \psi'_{i\eta}\psi'_{i\eta}\,d\zeta d\eta
\]

We can see from this that the resulting matrix will be symmetric, let's
consider the case when $i = 1 = j$:

\begin{align*}
    a\int_T(\psi'_{1\zeta})^2 + (\psi'_{1\eta})^2\,d\zeta d\eta & =
        a\int_T(-1)^2 + (-1)^2\,d\zeta d\eta \\
    &= 2a\int_0^1\int_0^\eta\,d\zeta d\eta \\
    &= 2a\frac{1}{2} \\
    &= a
\end{align*}

Following a similar process for each entry we find that this term gives the
matrix:

\begin{equation}\label{eq:stiffness_term_1}
    a\left(\begin{array}{c c c}
        1      & -\half & -\half \\
        -\half & \half  & 0 \\
        -\half & 0      & \half
    \end{array}\right)
\end{equation}

Now let's consider the second term, in the case when $i = 2, j = 1$:

\begin{align*}
    h^2\int_T{\v{b}\psi_1\nabla\psi_2}\, d\zeta d\eta &=
    -h\int_0^1\int_0^\eta{\zeta b_x + \zeta b_y}\, d\zeta d\eta \\
    &= -h(b_x + b_y)\int_0^1\int_0^\eta{\zeta}\, d\zeta d\eta \\
    &= -\frac{h(b_x + b_y)}{6}
\end{align*}

Repeating this for the other entries we get the matrix:

\begin{equation}\label{eq:stiffness_term_2}
    h\left(\begin{array}{c c c}
        \frac{b_x + b_y}{12} & -\frac{(b_x + b_y)}{6} & -\frac{(b_x + b_y)}{6} \\
        \frac{b_x}{12} & \frac{b_x}{4} & \frac{b_y}{6} \\
        \frac{b_y}{12} & \frac{b_x}{6} & \frac{b_y}{4}
    \end{array}\right)
\end{equation}

Finally let's consider the third integral, in the case where $i = j = 1$:

\begin{align*}
       ch^2\int_T{\psi_1\psi_1}\,d\zeta d\eta
       &= ch^2\int_0^1\int_0^\eta{(1 - \zeta - \eta)^2}\, d\zeta d\eta \\
%
       &= ch^2\int_0^1\int_0^\eta\, d\zeta d\eta +
          2ch^2\int_0^1\int_0^\eta{\zeta\eta - \zeta - \eta}\, d\zeta d\eta +
          ch^2\int_0^1\int_0^\eta{\zeta^2 + \eta^2}\, d\zeta d\eta \\
%
       &= \frac{ch^2}{2} - \frac{3ch^2}{4} + \frac{ch^2}{3} \\
       &= \frac{ch^2}{12}
\end{align*}

Following a similar argument for the other entries gives us the following
matrix:

\begin{equation}\label{eq:stiffness_term_3}
    \frac{ch^2}{2}\left(\begin{array}{c c c}
          \frac{1}{12} & - \frac{1}{24} & - \frac{1}{24} \\
        - \frac{1}{24} &   \frac{1}{12} & - \frac{1}{24} \\
        - \frac{1}{24} & - \frac{1}{24} &   \frac{1}{12}
    \end{array}\right)
\end{equation}

Combining (\ref{eq:stiffness_term_1}), (\ref{eq:stiffness_term_2}),
(\ref{eq:stiffness_term_3}) gives us the local stiffness matrix for the general
form of the equation (\ref{eq:general_laplace})

\section{The Local Mass Matrix}

The local mass matrix will also have dimension $3 \times 3$
